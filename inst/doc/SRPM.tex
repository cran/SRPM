\documentclass{article}

%%\VignetteIndexEntry{The SRPM Package}
%%\VignetteDepends{SRPM}

\usepackage{amsmath}
\usepackage{hyperref}
\usepackage{natbib}
\usepackage[margin=1.25in]{geometry}
\usepackage{charter}
\newcommand{\proglang}{\textsf}
\newcommand{\code}{\texttt}
\newcommand{\pkg}{\textbf}
\renewcommand{\baselinestretch}{1.1}

%%%%%%%%%%%%%%%%%%%%%%%%%%%%%%
%% declarations for jss.cls %%%%%%%%%%%%%%%%%%%%%%%%%%%%%%%%%%%%%%%%%%
%%%%%%%%%%%%%%%%%%%%%%%%%%%%%%

%% almost as usual
\author{Roger D. Peng\\Department of Biostatistics\\Johns Hopkins Bloomberg School of Public Health}

\title{The \pkg{SRPM} Package}

%% end of declarations %%%%%%%%%%%%%%%%%%%%%%%%%%%%%%%%%%%%%%%%%%%%%%%


\usepackage{/home/rpeng/install/R-2.5.0/lib64/R/share/texmf/Sweave}
\begin{document}

%% include your article here, just as usual
%% Note that you should use the \pkg{}, \proglang{} and \code{} commands.

\maketitle

\section{Introduction}

The \pkg{SRPM} package (``Shared Reproducibility Package Management'')
provides tools for creating and interacting with what we call ``shared
reproducibility packages''.  These shared reproducibility packages
(SRPs) are \textit{not} true \proglang{R} packages but rather contain
information related to a Sweave document that can be distributed to a
wide audience and used to reproduce the results.  The format of an SRP
is meant to be simple so that it can be used on different systems.

Each package is simply a directory which contains the following
subdirectories:
\begin{itemize}
\item
\code{article/}: contains the original Sweave file and a ``weaved''
version of the file (e.g. in PDF format)
\item
\code{figures/}: contains files corresponding to any figures in the
document
\item
\code{cacheDB/}: contains a collection of \pkg{stashR} databases
storing any cached computations from code chunks
\item
\code{code/}: contains code files corresponding to the \proglang{R}
code for each code chunk in the Sweave document (these are produced
with the \code{Stangle} function with the argument \code{split =
TRUE}).
\end{itemize}
Each SRP also has a file called \code{metadata.dcf} which is a text
file containing information about each of the code chunks in the
document.  This file is written in the Debian Control File format.

In addition, a package may contain a file named \code{REMOTE} which
contains the URL of the location of any remote \pkg{stashR} databases
containing cached computations.  If the size of the cached computation
databases is large, an author may wish to post them on a webserver
rather than distribute them with the SRP itself.  The \code{REMOTE}
file indicates the location of the cached computation databases and
configures the other tools in the \pkg{SRPM} package to retrieve data
from this location using the functionality in the \pkg{stashR}
package.


\section{Author tools}

A shared reproducibility package can be constructed with the
\code{makeSRP} function from the \pkg{SRPM} package which takes as
arguments the name of the package to create and the name of the Sweave
file for the original document.  The \code{makeSRP} function
\begin{enumerate}
\item
creates the necessary directories for the SRP;
\item
calls \code{Stangle} to create individual code files for each code
chunk and copies the files into the \code{code/} subdirectory;
\item
copies the \pkg{stashR} databases containing the cached computations
into the \code{cacheDB/} subdirectory;
\item
copies graphics files corresponding to figures into the
\code{figures/} subdirectory;
\item
copies the article PDF file and Sweave file into the \code{article/}
subdirectory;
\item
creates the metadata file by reading the map file produced by the
\pkg{cacheSweave} package and writes it to the \code{metadata.dcf}
file.
\end{enumerate}
Currently, the \pkg{SRPM} package requires that both the graphics
files for the figures and the weaved version of the article be in PDF
format, however we hope to remove this limitation in the near future.

Another function that is available to authors is the
\code{makeWebpage} function which produces a simple webpage
corresponding to an SRP.  The webpage lists all of the code chunks in
a document with links to the code itself.  Also, there are links to
cache databases as well as the PDF versions of figures so that readers
can browse an SRP using a web browser and without having to have R
installed.






\section{Reader tools}

Sweave is an example of a tool that is useful to authors of
statistical or scientific documents in that it assists in the
development of documents by ensuring that the text and data analysis
are closely integrated into a single document.  However, readers of
reproducible documents also need tools to assist them with interacting
with the data analyses therein and reproducing key results.

In the \pkg{SRPM} package we provide some basic tools for readers of
Sweave documents that allow them to interact with the code and data
provided by a shared reproducibility package created by the author.
The basic functions are
\begin{itemize}
\item
\code{code}: When called with no arguments, a listing of all the code
chunks in the article is printed to the console.  The \code{code}
function can also take a numeric argument corresponding to the code
chunk sequence number of a character argument corresponding to the
code chunk name.  When \code{code} is passed a numeric or character
argument, it returns an object of class ``codeObject'' which contains
the code and pointers to any cached computation databases or figures
associated with the code chunk.
\item
\code{article}: This function takes no arguments; when called it
launches the article PDF document in the PDF viewer.
\item
\code{figure}: This function must be given a numeric argument
corresponding to the figure number in the original article.  When
called, it displays in the PDF viewer the figure corresponding to the
figure number.
\item
\code{cache}: This function takes a code chunk sequence number or a
code chunk name (character) as an argument and returns an object of
class ``localDB'' or ``remoteDB'' depending on whether the SRP is
using local or remote cached computation databases.  This object can
be explored with the methods defined in the \pkg{stashR}
package~\citep[see also details in][]{ecke:peng:2006}.
\item
\code{loadcache}: This function lazy-loads cached computation
databases into the global environment.  It takes a numeric vector of
code chunk sequence numbers or a character vector of code chunk names
and loads the cached computation databases associated with those code
chunks in the order that they are specified.  Once a database is
lazy-loaded, the object names appear in the environment into which
the database was loaded, but they do not occupy any extra memory
until they are first accessed.  If a specified code chunk does not
have a database associated with it, no action is taken.
\item
\code{runcode}: The \code{runcode} function takes as input a numeric
vector of code chunk sequence numbers or a character vector of code
chunk names and executes the code in those code chunks.  Each code
chunk is evaluated in the order in which it appears in the input
vector.  By default, if a cached computation database is associated
with a code chunk, then the database is lazy-loaded via
\code{loadcache} rather than executed.  In order to force evaluation
of code in a code chunk with a cache database, one needs to set
\code{useCache = FALSE} when calling \code{runcode}.  If an error
occurs when executing the code in a code chunk, a message is printed
to the console indicating the error and the code chunk is skipped.
\item
\code{edit}: A method is provided for the \code{edit} generic function
for objects of class ``codeObject'' which can be used to edit the
\proglang{R} code corresponding to a code chunk.  The modified
``codeObject'' object can be executed with the \code{runcode}
function.  The \code{edit} method returns the modified copy of the
object so that the original code is not modified.  The editor used is
that which is launched by the \code{file.edit} function and will be
system dependent.
\end{itemize}

These functions consist of the primary user interface for readers to
interact with shared reproducibility package.  Certain SRPs may also
require that other \proglang{R} packages be installed in order to
execute the code in the code chunks and these should be installed
before attempting to execute the code with \code{runcode}.

Other utility functions available to the user are
\code{currentPackage}, which shows the currently registered SRP,
\code{getRemoteURL}, which returns the URL of the remote cached
computation databases (if any), and \code{getLocalDir}, which returns
the path to the directory where local copies of the remote cache
databases will be stored.







\section{Example}
\label{sec:example}

The \pkg{SRPM} package depends on the \pkg{methods} and \pkg{stashR}
packages and additionally imports the \pkg{utils}, \pkg{filehash}, and
\pkg{cacheSweave} packages.  Once those dependencies are installed,
the \pkg{SRPM} package can be loaded using \code{library} in the usual
way.

The first thing a user must do is register a shared reproducibility
package (SRP) using the \code{setPackage} function.  We will use as an
example a simple SRP called \code{srp\_simple} that comes with the
\pkg{SRPM} package.  The package can be registered by passing the name
of the directory to the \code{setPackage} function.
\begin{Schunk}
\begin{Sinput}
> library(SRPM)
> pkgdir <- system.file("SRP-ex", "srp_simple", package = "SRPM")
> setPackage(pkgdir)
\end{Sinput}
\end{Schunk}
Upon registering a package one can call the \code{article} function
(with no arguments) to open a PDF copy of the full article in the PDF
viewer (as identified by \code{getOption("pdfviewer")}).  Another
useful function to begin with is the \code{code} function.  Called
with no arguments, \code{code} lists all of the code chunks in the
article.
\begin{Schunk}
\begin{Sinput}
> code()
\end{Sinput}
\begin{Soutput}
1 LoadPackages  
2 FitLinearModel  [C]
3 CoefficientTable  
4 LinearModelDiagnosticPlots [Figure 1] 
\end{Soutput}
\end{Schunk}

The code chunk listing is annotated with three different types of
tags.  The first is the code chunk sequence number which appears to
the left of the code chunk name.  This number can be used to identify
a code chunk in other operations.  The second is a \texttt{[C]} which
indicates caching has been turned on and that the corresponding code
chunk gives rise to a cached computation database.  Lastly, code
chunks with the tag \texttt{[Figure ?]} produce figures or plots.  For
example, code chunk 4 produces Figure 1 in the original article.

%% runcode example

Code in any of the code chunks can be executed with the \code{runcode}
function by passing the code chunk name or sequence number.  Sequences
of code chunks can be executed by passing a numeric or character
vector to \code{runcode}.  For example, to execute the code in chunks
1 through 3 to create the table of regression coefficients, we can
execute
\begin{Schunk}
\begin{Sinput}
> runcode(1:3)
\end{Sinput}
\end{Schunk}
which prints the following messages
\begin{verbatim}
running code in code chunk 1
loading cache for code chunk 2
running code in code chunk 3
\end{verbatim}
and prints a table to the console (here, using the \pkg{xtable}
package).  Code chunk 2 has a cached computation database associated
with it so the database is lazy-loaded into the workspace instead of
the code being executed.  Code chunk 3 creates the table and it is
executed successfully.

%% loadcache example

In order to explicitly lazy-load a cached computation database into
the workspace one can use the \code{loadcache} function.  For example,
to load the database for code chunk 2, one can call
\begin{Schunk}
\begin{Sinput}
> loadcache(2)
> ls()
\end{Sinput}
\begin{Soutput}
[1] "airquality" "fit"        "pkgdir"    
\end{Soutput}
\end{Schunk}

The ``localDB'' object representing the cached computation database
can be explicitly retrieved by calling the \code{cache} function and
accessed using the methods defined in the \pkg{stashR} package.  The
above code fragment is roughly equivalent to
\begin{Schunk}
\begin{Sinput}
> library(stashR)
> db <- cache(2)
> show(db)
\end{Sinput}
\begin{Soutput}
'localDB' database 'FitLinearModel'
\end{Soutput}
\begin{Sinput}
> dbList(db)
\end{Sinput}
\begin{Soutput}
[1] "airquality" "fit"       
\end{Soutput}
\begin{Sinput}
> dbLazyLoad(db)
\end{Sinput}
\end{Schunk}


%% \section{Remote packages}
%% 
%% Shared reproducibility packages can be structured in such a way that
%% the cached computation databases can be stored on a remote server and
%% accessed over the web using \proglang{R}'s Internet capabilities.  In
%% that case the SRP that is distributed does not have any of the
%% \pkg{stashR} databases in it but rather contains a \code{REMOTE} file
%% containing the URL indicating the location of the cached computation
%% databases on the web.  A remote version of the \code{srp\_simple}
%% package called \code{srp\_simple\_R} is available in the \pkg{SRPM}
%% package and can be loaded by calling \code{setPackage}.
%% <<LoadRemoteSRPSimple>>=
%% pkgdir <- system.file("SRP-ex", "srp_simple_R", package = "SRPM")
%% @ 
%% <<setPackageRemote,eval=false>>=
%% setPackage(pkgdir)
%% @ 
%% The \code{REMOTE} file for this package contains the following information:
%% <<loadRemoteSRP>>=
%% remote <- readLines(file.path(pkgdir, "REMOTE"))
%% writeLines(remote)
%% @
%% 
%% The \code{RemoteURL} field in the file contains the URL for the cached
%% computation databases.  If a \code{REMOTE} file exists it is
%% automatically read by the \code{setPackage} function.  The remote URL
%% can be retrieved in an \proglang{R} session via the
%% \code{getRemoteURL} function.  Once a remote package is registered
%% with \code{setPackage} all of the functions mentioned above will work,
%% however, depending on the speed of the network connection there may be
%% some delay in downloading large objects when they are accessed.

%% As a convenience, \code{setPackage} can download and unzip the SRP for
%% you, provided a suitable \code{unzip} program is available.  For
%% example, running
%% <<getRemoteZipFile,eval=false>>=
%% setPackage("http://www.biostat.jhsph.edu/~rpeng/RR/srp_seasonal_R.zip")
%% @ 
%% will download and unzip the SRP as well as register it as the
%% current package


\section{Package websites}

If the author has created a webpage via the \code{makeWebpage}
function, the reader may prefer to view that first.  The webpage
created by \code{makeWebpage} corresponding to the
\code{srp\_simple} package can be found at
\begin{verbatim}
http://www.biostat.jhsph.edu/~rpeng/RR/ea6883e28967ec48a0ee009585da0eb0/html/
\end{verbatim}
Currently, the webpage created by \code{makeWebpage} resembles the
output provided by the \proglang{R} functions at the console.  The
cached computation databases can be browsed in a limited
fashion---smaller objects can be viewed in their entirety while for
larger objects a summary is provided by showing the output from
\code{str}.





\bibliography{combined}
\bibliographystyle{jss}


\end{document}
